\documentclass[11pt,a4paper]{amsart}
\usepackage{setspace}
\doublespacing
\usepackage{amssymb,latexsym}
\usepackage{graphicx}
\theoremstyle{plain}
\newtheorem{theorem}{Theorem}
\newtheorem{corollary}{Corollary}
\newtheorem{lemma}{Lemma}
\newtheorem{axiom}{Axiom}
\newtheorem{proposition}{Proposition}
\usepackage{geometry}
\geometry{a4paper,left=2cm,right=2cm,top=1cm,bottom=1cm}
\theoremstyle{definition}
\newtheorem{definition}{Definition}
% \usepackage{ulem} % various underlines
\usepackage{hyperref} % to insert URL 
\usepackage{graphicx} % to insert illustration
\usepackage[mathscr]{eucal} % to express a collection of sets
\usepackage{bm} % bold font in equation environment
\usepackage{color} % color some text
\usepackage{framed} % to add a frame 
\usepackage{tikz}
\newcommand*\circled[1]{\tikz[baseline=(char.base)]{
		\node[shape=circle,draw,inner sep=1pt] (char) {#1};}} % circled numbers
\usepackage{float}%do not auto repositioning
% $\uppercase\expandafter{\romannumeral1}$ Roman numeral
%	\begin{figure}[hbt]
	%{\centering \includegraphics[scale=0.78]{ring_algebra_semi}}
	%\caption{ring \& algebra \& semi-}\label{F:ring_algebra_semi}
	%\end{figure}
\usepackage[style=apa, eprint=false]{biblatex} %Imports biblatex package
\addbibresource{specms.bib} %Import the bibliography file
	
\begin{document}
\title{Hypothesis tests}
\date{\today}
\maketitle
\tableofcontents
\newpage
		
\section{Specification tests}
\subsection{Likelihood ratio tests for fixed effects. }\hfill\par
		
\textit{The likelihood ratio test} compares the log-likelihood of the unrestricted model with that of the restricted mode. The rationale is that if the null hypotheses are true, then these values should be close. The likelihood ratio test statistics is 
\[	-2 \ln \lambda = 2(\ln L(\hat{\theta}_{U}) - \ln L(\hat{\theta}_{R}) ),	\]
where $\hat{\theta}_{U}$ is the maximum likelihood estimate of the \emph{unrestricted} model, $\hat{\theta}_{R}$ is the maximum likelihood estimate of the \emph{restricted} model, and $\lambda$ is the likelihood ratio, defined as $\lambda = L(\hat{\theta}_{R}) / L(\hat{\theta}_{U})$.
		
This test statistic is distributed $\chi^{2}$ with degrees of freedom equal to the number of model restrictions. That is,
	\[	-2 \ln \lambda \sim \chi^{2}(J).	\]
		
$\bullet$ \textbf{$H_{0}:$ Spatial fixed effects are not jointly significant.} 
		
There are $283$ units in our data, the test statistic is distributed  $-2 \ln \lambda \sim \chi^{2}(282).$
		
The null hypothesis is 
\[	H_{0}: \begin{pmatrix}
		\mu_{1} \\
		\dots \\
		\mu_{283}
\end{pmatrix} = \begin{pmatrix}
		0 \\
		\dots \\
		0
\end{pmatrix}.	\]
		
The test statistic takes the value $6069.595$, which is much larger than the critical value for a $99.9\%$ confidence interval ($361.1201$). We accept the alternative hypothesis,  with a p-value less than $0.001$.

$\bullet$ \textbf{$H_{0}:$ Time fixed effects are not jointly significant.} 
		
There are $8$ time periods in our data, the test statistic is distributed  $-2 \ln \lambda \sim \chi^{2}(7).$
		
The null hypothesis is 
\[	H_{0}: \begin{pmatrix}
		\lambda_{1} \\
		\dots \\
		\lambda_{8}
\end{pmatrix} = \begin{pmatrix}
		0 \\
		\dots \\
		0
\end{pmatrix}.	\]
		
The test statistic takes the value $74.08778$, which is much larger than the critical value for a 99.9\% confidence interval ($24.32189$). We accept the alternative hypothesis, with a p-value less than $0.001$.
		
These results justify the usage of individual and time fixed effects. 
		
\subsection{Lagrange multiplier tests for spatial effects.}\footnote{Tests for spatial dependence in panel models by \parencite{anselinSpatialPanelEconometrics2008}.}\hfill\par
		
\subsection{Normality, serial independence, and homoskedasticity of the regression residuals}

Our maximum likelihood estimation depends on the assumption that $\epsilon_{it}$ is i.i.d. normally distributed. We can get some clues about this assumption by looking at the residuals. Here I am using the residuals of the two-regime regression and 3nn weighting matrix. 

By plotting the kernel density of the residuals, it seems like that we have a higher-than-normal kurtotics. It is pretty symmetric, so it is not ``skewed'', but it has heavy tails.

\begin{figure}[hbt]
	{\centering \includegraphics[scale=0.78]{/Users/jialiangchen/Documents/spmodeltoruism/results/resid_plot.pdf}}
	\caption{Kernel density of the residual}\label{F:kden_resid}
\end{figure}

This is the Q-Q plot of the standardized residuals.
\begin{figure}[hbt]
	{\centering \includegraphics[scale=0.78]{/Users/jialiangchen/Documents/spmodeltoruism/results/qq_residuals.pdf}}
	\caption{Q-Q plot of the residual}\label{F:qq_residuals}
\end{figure}

If the residual is actually distributed normally, the dots should fit the red line well. It is argued that with a large sample, it is more important to focus on the standardized residuals within the range $[-2, 2]$. This is because these tests are tend to over-reject the null when  sample size is large.  In the tails, due to lower density, deviations (from the normal distribution) are expected but not of concern. As you can see, most of our residuals within $[-2, 2]$ fits the line quite well. Only a limited number of them outside that range deviates from the red line.

We can also test this formally. But this kind of test should be taken with a grain of salt.\footnote{See \href{https://stats.stackexchange.com/questions/2492/is-normality-testing-essentially-useless}{here} for a discussion on this issue.} We use Kolmogorov–Smirnov test for the test of normality. If we focus on this range (with $2141$ out of $2264$ observations), we have the following test results.

$\bullet$ \textbf{$H_{0}:$ The econometric residuals are drawn from a normal distribution.} 

The Kolmogorov–Smirnov test statistic is  $0.0258$ with a p-value of $0.1158$. There is not enough evidence to reject the assumption of normally distributed residuals. 



\section{Other tests}

\printbibliography

\end{document}
